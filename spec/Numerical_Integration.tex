\sekshun{Numerical Integration}
\label{Numerical_Integration}
\index{numerical integration}

\section{Rectangle Method}
The rectangle method computes an approximation to a 
definite integral by finding the area of a collection of rectangles whose heights are determined 
by the values of the function.  Specifically, the interval $[a,b]$ over which the function is to 
be integrated is divided into $N$ equal subintervals of length $h = (b-a)/N$. The rectangles are 
drawn with one base along the $x$-axis. Depending on whether the method is left, right, or midpoint,
the left corner, right corner, or midpoint, respectively, of the side opposite the base lies on the 
graph of the function. The approximation to the integral is 
then calculated by adding up the areas (base multiplied by height) of the $N$ rectangles, giving the formula:
\begin{equation}
\int_a^b f(x) dx \approx h \sum_{n=0}^{N-1} f(x_n) \label{eq:rectangle}
\end{equation}
where
\begin{equation}
h=(b-a)/N 
\end{equation}

\begin{table}[htbp]
\centering
\caption{Formula for $x_n$ in Equation \ref{eq:rectangle} of rectangle numerical integration methods.} \label{tab:xn-rectangle}
\begin{tabular}{cc}
\textbf{Method} & \textbf{$x_n$} \\ \toprule
left & $a+nh$ \\ \midrule
right & $a+(n+1)h$ \\ \midrule
midpoint & $a+\left(n + \frac{1}{2}\right)h$ \\ \bottomrule
\end{tabular}
\end{table}

The formula for $x_n$ for the left, right, and midpoint methods are given in Table \ref{tab:xn-rectangle}.
As $N$ gets larger, the rectangle method becomes more accurate. This is illustrated in the series of plots
in Figure \ref{fig:rectangle}.

\begin{figure}
\centering
\subcaptionbox{$N=4$}{\includegraphics[scale=.6]{fig/rightrectangle-4.eps}}
\subcaptionbox{$N=6$}{\includegraphics[scale=.6]{fig/rightrectangle-6.eps}}
\subcaptionbox{$N=10$}{\includegraphics[scale=.6]{fig/rightrectangle-10.eps}}
\caption{Numerical integration of $f(x) = (2 x-0.5)^3+(1.5 x-1)^2-x+1$ for $x$ in $[0.1,0.8]$
by the (right) rectangle method for increasing values of $N$. The number inside each rectangle is
the area of that rectangle, and the total area is displayed on each graph.
The exact value of the integral is 0.70525.}\label{fig:rectangle}
\end{figure}

If $f(x)$ is increasing or decreasing on the interval $[a,b]$, the maximum error $E$ 
for left or right rectangular numerical integration is given by
\begin{equation}
E \leq \frac{b-a}{N}\left|f(b)-f(a)\right| \label{eq:lr-rectangle-max-error}
\end{equation}

We can create a helper function to compute the maximum error for left and right rectangle
methods using \ref{eq:lr-rectangle-max-error}. The calculated value will be used
in tests for the left and right rectangle methods to check that the result is within 
the maximum error expected for a given $a$, $b$, and $N$. 
\begin{enumspec}
\item\spec{1} Helper function \lstinline{leftRightRectangleMaxErr} returns the
maximum error expected for left or right rectangle method numerical integration. It 
takes in a reference to a pre-defined function $f$, the bounds $a$ (real) and $b$ (real) of the 
interval for definite integration, and the number $N$ (integer) of subintervals used.
The function will be entered in \lstinline{leftRightRectangleMaxErr.chpl}.
\meetsrequirement{5}
\end{enumspec}

\begin{chapelhelper}{leftRightRectangleMaxErr.chpl}
\begin{chapel}
proc leftRightRectangleMaxErr(a: real, b: real, N: int, f){
  return ((b-a)/N)*abs(f(b)-f(a));
}
\end{chapel}
\end{chapelhelper}

One of the functions that we need to test our methods against is $f(x) = x^3$, 
with $a=0$, $b=1$, and $N=100$.
Since the function is increasing on the interval $[0,1]$, we can use 
the helper function that we just created to compute the maximum expected error. We are
ready to create our first test for a function that we will write to compute the definite
integral using the left rectangle method. This function will be called \lstinline{leftRectangleIntegration}
and will be written to \lstinline{leftRectangleIntegration.chpl}.
\begin{enumspec}
\item\spec{2}
Test \lstinline{leftRectangleIntegrationTest.chpl} loads modules
\lstinline{leftRightRectangleMaxErr} and
\lstinline{leftRectangleIntegration}.
It defines a function \lstinline{f} that takes $x$ (real) and returns $x^3$ (real).
It passes $a=0.0$, $b=1.0$, $N=100$, and \lstinline{f} to the function
\lstinline{leftRightRectangleMaxErr} and stores the result in the variable
\lstinline{maximumError} (real).
It passes $a=0.0$, $b=1.0$, $N=100$, and \lstinline{f} to the function
\lstinline{leftRectangleIntegration} and stores the result in the variable
\lstinline{calculated}.
It then checks to see if the absolute value of the difference between \lstinline{calculated} 
and \lstinline{exact} is less than or equal to \lstinline{maximumError} and sets 
\lstinline{verified} (bool). The test writes out \lstinline{verified} and a passing
test results in \lstinline{true}.
\meetsrequirement{5.1}
\end{enumspec}

\begin{chapelexample}{leftRectangleIntegrationTest.chpl}
A test for \lstinline{leftRectangleIntegration}.
\begin{chapelpre}
use leftRightRectangleMaxErr;
use leftRectangleIntegration;
\end{chapelpre}
\begin{chapel}
proc f(x:real):real {
  return x**3;
} 
  
var calculated:real;
var exact:real = 0.25;  // from Mathematica
var maximumError:real = leftRightRectangleMaxErr(a = 0.0, b = 1.0, N = 100, f = f);
var verified:bool;

calculated = leftRectangleIntegration(a = 0.0, b = 1.0, N = 100, f = f);
verified = (abs(calculated - exact) <= maximumError);
writeln(verified);
\end{chapel}
\begin{chapelpost}
\end{chapelpost}
\begin{chapeloutput}
true
\end{chapeloutput}
\end{chapelexample}

Now that we have our first test, it's time to verify that it does not pass.
\begin{TODO}
Describe the notation for command prompt. Describe output for each of the commands below. 
Fix "make test" so you don't have to go into the test directory. Create a make all. Create
an environment for the stuff that's not actually part of the spec and is meant to show what
the spec generates.
\end{TODO}
\begin{verbatim}
[~/chapel/integrate/] $ make spectests
[~/chapel/integrate/] $ make specsources
[~/chapel/integrate/] $ cd test
[~/chapel/integrate/test/] $ testseamless
\end{verbatim}

The code that provides the \lstinline{leftRectangleIntegration} function is straightforward:
\begin{chapelsource}{leftRectangleIntegration.chpl}
\begin{chapel}
proc leftRectangleIntegration(a: real(64), b: real(64), N: int(64), f): real(64){
  var h: real(64) = (b - a)/N;
  var sum: real(64) = 0.0;
  var x: real(64);
  for n in 0..N-1 {
    x = a + n * (b-a)/N;
    sum = sum + f(x);
  }
  return h * sum;
}
\end{chapel}
\end{chapelsource}

For a function $f$ which is twice differentiable, the maximum error $E$ is given by
the following equation:
\begin{equation}
E \leq \frac{(b-a)h^2}{24} f''(\xi) \label{eq:rectangle-max-error}
\end{equation}
for some $\xi$ in $[a,b]$.

Create a helper function to compute the maximum error:
\begin{chapelhelper}{midpointRectangularIntegrationMaximumError.chpl}
\begin{chapel}
proc midpointRectangularIntegrationMaximumError(a: real, b: real, N: int, fppxi){
  var h:real = (b-a)/N;
  return ((b-a)*h**2/24) * fppxi;
}
\end{chapel}
\end{chapelhelper}
