\sekshun{Notation}
\label{Notation}
\index{notation}

Three types of code chunks that make up the software package are 
presented in this specification and delineated
with the appropriate keyword in italics: 
\textit{Sources, Helpers, and Tests}.  The filename containing the code
chunk is given in parenthesis following the keyword. A brief description 
of the code chunk is also listed.
The actual code is represented with a fixed-width font where keywords are
bold and comments are italicized. An example helper code chunk is listed
below.

\textit{Helper (testFunctions.chpl)}. Provide the functions used
in the tests.
\begin{chapel}
  proc f1(x:real):real {
    return x**3;
  } 
  proc f2(x:real):real {
    return 1/x;
  } 
  proc f3(x:real):real {
    return x;
  } 
\end{chapel}

Examples of how to use the software are also provided and delineated with the
keyword \textit{Example} followed by a description of the use case.

\begin{TODO}
  Add example.
\end{TODO}

Different color text boxes are used to highlight various types of items 
throughout the specification.

\begin{TODO}
  Things that need to be done for this version of the software.
\end{TODO}

\begin{note}
  Something of note that does not fit into any other category.
\end{note}

\begin{rationale}
  An explanation for a particular design choice.
\end{rationale}

\begin{openissue}
  Issue that we do not know how to handle.
\end{openissue}

\clearpage
\begin{future}
  Issue or feature that we have a story about, but which is not yet
  fully-designed or implemented. 
\end{future}
