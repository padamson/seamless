\sekshun{Simpson's Rule Integration}
\label{Simpsons_Integration}
\index{Simpson's rule integration}
\index{integration!Simpson's rule}

The Simpson's rule method approximates the function with a quadratic. The particular flavor that
we are going to use here requires that the interval $[a,b]$ is subdivided into an even number of
subintervals of width $h=(b-a)/N$. 
The general Simpson's rule is given by
\begin{equation}
  \int_a^b f(x) dx \approx \frac{h}{3} \left[ f(a) + 
  4\sum_{\substack{n=1\\n \text{odd}}}^{N-1} f(x_n) + 
  4\sum_{\substack{n=2\\n \text{even}}}^{N-2} f(x_n) + 
  f(b) \right] \label{eq:simpsons}
\end{equation}
where $x_n$ is still given by Equation~\ref{eq:xn-trapezoid} and 
$h$ is still given by Equation~\ref{eq:subinterval-width}.

For a function $f$ which has a fourth derivative, the maximum error $E$ for the
Simpson's rule method is given by the following equation:
\begin{equation}
  E \leq \frac{(b-a)^5}{180 N^4} f^{(4)}(\xi) \label{eq:simpsons-max-error}
\end{equation}
for some $\xi$ in $[a,b]$.

The expected error for the Simpson's rule method for all of the tests is expected  
to be very low, so we will use a value of 0.00001 for all of them.

\begin{chapelexample}{simpsonsIntegrationTest.chpl}
  A test for \lstinline{simpsonsIntegration} using $f(x) = \{x^3, 1/x, x\}$.
  \begin{chapelpre}
  \end{chapelpre}
  \begin{chapel}
    use simpsonsIntegration;
    use testFunctions;

    var exact:real;
    var calculated:real;
    var maxErr:real;

    exact = 0.25;
    maxErr = 0.00001;
    calculated = simpsonsIntegration(a = 0.0, b = 1.0, N = 100, f = f1);
    writeln((abs(calculated - exact) <= maxErr));

    exact = 4.605170;
    maxErr = 0.00001;
    calculated = simpsonsIntegration(a = 1.0, b = 100.0, N = 1000, f = f2);
    writeln((abs(calculated - exact) <= maxErr));

    exact = 12500000;
    maxErr = 0.00001;
    calculated = simpsonsIntegration(a = 0.0, b = 5000.0, N = 5000000, f = f3);
    writeln((abs(calculated - exact) <= maxErr));

    exact = 18000000;
    maxErr = 0.00001;
    calculated = simpsonsIntegration(a = 0.0, b = 6000.0, N = 6000000, f = f3);
    writeln((abs(calculated - exact) <= maxErr));
  \end{chapel}
  \begin{chapelpost}
  \end{chapelpost}
  \begin{chapeloutput}
true
true
true
true
  \end{chapeloutput}
\end{chapelexample}

\begin{chapelsource}{simpsonsIntegration.chpl}
  \begin{chapel}
    proc simpsonsIntegration(a: real(64), b: real(64), N: int(64), f): real{
      var h: real(64) = (b - a)/N; 
      var sum: real(64) = f(a) + f(b);
      var x_n: real(64);
      for n in 1..N-1 by 2 {
        x_n = a + n * h;
        sum = sum + 4.0 * f(x_n);
      }
      for n in 2..N-2 by 2 {
        x_n = a + n * h;
        sum = sum + 2.0 * f(x_n);
      }
      return (h/3.0) * sum;
    }
  \end{chapel}
\end{chapelsource}

\begin{chapelexample}{simpsonsIntegrationParallelTest.chpl}
  A test for \lstinline{simpsonsIntegrationParallel} using $f(x) = \{x^3, 1/x, x\}$.
  \begin{chapelpre}
  \end{chapelpre}
  \begin{chapel}
    use simpsonsIntegrationParallel;
    use testFunctions;

    var exact:real;
    var calculated:real;
    var maxErr:real;

    exact = 0.25;
    maxErr = 0.00001;
    calculated = simpsonsIntegrationParallel(a = 0.0, b = 1.0, N = 100, f = f1);
    writeln((abs(calculated - exact) <= maxErr));

    exact = 4.605170;
    maxErr = 0.00001;
    calculated = simpsonsIntegrationParallel(a = 1.0, b = 100.0, N = 1000, f = f2);
    writeln((abs(calculated - exact) <= maxErr));

    exact = 12500000;
    maxErr = 0.00001;
    calculated = simpsonsIntegrationParallel(a = 0.0, b = 5000.0, N = 5000000, f = f3);
    writeln((abs(calculated - exact) <= maxErr));

    exact = 18000000;
    maxErr = 0.00001;
    calculated = simpsonsIntegrationParallel(a = 0.0, b = 6000.0, N = 6000000, f = f3);
    writeln((abs(calculated - exact) <= maxErr));
  \end{chapel}
  \begin{chapelpost}
  \end{chapelpost}
  \begin{chapeloutput}
true
true
true
true
  \end{chapeloutput}
\end{chapelexample}

\begin{chapelsource}{simpsonsIntegrationParallel.chpl}
  \begin{chapel}
    proc simpsonsIntegrationParallel(a: real(64), b: real(64), N: int(64), f): real{
      var h: real(64) = (b - a)/N; 
      var sum1$, sum2$: sync real = 0.0;
      var x_n1$, x_n2$: sync real;
      cobegin {
        for n1 in 1..N-1 by 2 {
          x_n1$ = a + n1 * h;
          sum1$ = sum1$ + 4.0 * f(x_n1$);
        }
        for n2 in 2..N-2 by 2 {
          x_n2$ = a + n2 * h;
          sum2$ = sum2$ + 2.0 * f(x_n2$);
        }
      }
      return (h/3.0) * (f(a) + sum1$ + sum2$ + f(b));
    }
  \end{chapel}
\end{chapelsource}

\begin{TODO}
  Fix chapel\_listing.tex to handle \$ characters for sync variables.
\end{TODO}

\begin{chapelexample}{simpsonsIntegrationParallelTestWithTiming.chpl}
  A test for \lstinline{simpsonsIntegration} and \lstinline{simpsonsIntegrationParallel} 
  using $f(x) = \{x^3, 1/x, x\}$ comparing timing of the parallel and serial methods.
  \begin{chapelpre}
  \end{chapelpre}
  \begin{chapel}
    use simpsonsIntegrationParallel;
    use simpsonsIntegration;
    use testFunctions;
    use Time;

    var exact:real;
    var calculated:real;
    var maxErr:real;
    var timer, timerP:Timer;
    var timerMargin: real = 0.6;

    exact = 0.25;
    maxErr = 0.00001;
    timerP.start();
    calculated = simpsonsIntegrationParallel(a = 0.0, b = 1.0, N = 100, f = f1);
    timerP.stop();
    writeln((abs(calculated - exact) <= maxErr));
    timer.start();
    calculated = simpsonsIntegration(a = 0.0, b = 1.0, N = 100, f = f1);
    timer.stop();
    writeln((abs(calculated - exact) <= maxErr));
    writeln((timerP.elapsed() < timerMargin*timer.elapsed()));

    exact = 4.605170;
    maxErr = 0.00001;
    timerP.clear();
    timerP.start();
    calculated = simpsonsIntegrationParallel(a = 1.0, b = 100.0, N = 1000, f = f2);
    timerP.stop();
    writeln((abs(calculated - exact) <= maxErr));
    timer.clear();
    timer.start();
    calculated = simpsonsIntegration(a = 1.0, b = 100.0, N = 1000, f = f2);
    timer.stop();
    writeln((abs(calculated - exact) <= maxErr));
    writeln((timerP.elapsed() < 0.5*timer.elapsed()));

    exact = 12500000;
    maxErr = 0.00001;
    timerP.clear();
    timerP.start();
    calculated = simpsonsIntegrationParallel(a = 0.0, b = 5000.0, N = 5000000, f = f3);
    timerP.stop();
    writeln((abs(calculated - exact) <= maxErr));
    timer.clear();
    timer.start();
    calculated = simpsonsIntegration(a = 0.0, b = 5000.0, N = 5000000, f = f3);
    timer.stop();
    writeln((abs(calculated - exact) <= maxErr));
    writeln((timerP.elapsed() < timerMargin*timer.elapsed()));

    exact = 18000000;
    maxErr = 0.00001;
    timerP.clear();
    timerP.start();
    calculated = simpsonsIntegrationParallel(a = 0.0, b = 6000.0, N = 6000000, f = f3);
    timerP.stop();
    writeln((abs(calculated - exact) <= maxErr));
    timer.clear();
    timer.start();
    calculated = simpsonsIntegration(a = 0.0, b = 6000.0, N = 6000000, f = f3);
    timer.stop();
    writeln((abs(calculated - exact) <= maxErr));
    writeln((timerP.elapsed() < timerMargin*timer.elapsed()));
  \end{chapel}
  \begin{chapelpost}
  \end{chapelpost}
  \begin{chapeloutput}
true
true
true
true
true
true
true
true
true
true
true
true
  \end{chapeloutput}
\end{chapelexample}
